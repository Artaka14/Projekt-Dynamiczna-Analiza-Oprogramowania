\documentclass[12pt,a4paper]{article}
\usepackage[T1]{fontenc}
\usepackage[utf8]{inputenc}
\usepackage[polish]{babel}
\usepackage{lmodern}
\usepackage{float}
\usepackage{graphicx}
\usepackage{hyperref}
\usepackage{geometry}
\geometry{margin=2.5cm}

\title{XD PROJEKT – Dokumentacja aplikacji}
\author{Dariusz Kołodziejczyk, Mikołaj Maliszewski}

\begin{document}
	\maketitle
	
	\section{Krótki opis}
	Aplikacja służy do śledzenia cen akcji CD PROJEKT S.A. w wybranych okresach czasu, umożliwia prezentację wykresów oraz bieżącej ceny papierów wartościowych.
	
	\section{Cele i założenia projektu}
	
	Głównym celem aplikacji jest stworzenie narzędzia umożliwiającego użytkownikowi:
	
	\begin{itemize}
		\item bieżące monitorowanie kursu akcji CD Projekt S.A.,
		\item analizę zmian wartości w wybranych okresach czasu,
		\item zestawienie kursu z poziomem zainteresowania w Internecie (Google Trends),
		\item pobieranie podstawowych dokumentów finansowych udostępnianych przez spółkę.
	\end{itemize}
	
	\section{Środowisko i wymagania}
	
	Aplikacja została stworzona w języku Python w wersji 3.11.9 i działa w środowisku graficznym opartym na bibliotece CustomTkinter.
	
	Do prawidłowego działania wymagane są następujące biblioteki:
	
	\begin{itemize}
		\item \verb|yfinance| – pobieranie danych giełdowych z Yahoo Finance,
		\item \verb|matplotlib| – generowanie wykresów kursu,
		\item \verb|Pillow| – obsługa grafiki,
		\item \verb|customtkinter| – tworzenie nowoczesnego interfejsu użytkownika,
		\item \verb|CTkMessagebox| – obsługa okien komunikatów,
		\item \verb|tkcalendar| – wybór zakresu dat,
		\item \verb|pytrends| – komunikacja z Google Trends.
		\item \verb|openpyxl| - otwieranie plików XSLX.
		\item \verb|pandas| - otwieranie i działanie na plikach XSLX.
		\item \verb|pytest| - biblioteka obsługująca testy
	\end{itemize}
	
	Instalacja pakietów za pomocą pliku req.txt:
	\begin{verbatim}
		pip install -r req.txt
	\end{verbatim}
	
	\section{Opis funkcjonalności}
	Aplikacja realizuje następujące funkcje:
	
	\begin{itemize}
		\item pobieranie aktualnej ceny akcji CD Projekt S.A.
		\item wyświetlanie wykresów kursu z zakresu:
		\begin{itemize}
			\item ostatniego dnia (\verb|1d|)
			\item ostatniego tygodnia (\verb|7d|)
			\item ostatniego miesiąca (\verb|1m|)
			\item dowolnego zakresu wybranego z kalendarza
		\end{itemize}
		\item pobieranie i prezentacja wykresów z Google Trends z zakresu:
		\begin{itemize}
			\item ostatniego dnia (\verb|1d|)
			\item ostatniego tygodnia (\verb|7d|)
			\item ostatniego miesiąca (\verb|1m|)
		\end{itemize}
		\item możliwość wyboru słowa kluczowego (np. CD Projekt'', Cyberpunk 2077'', ``Wiedźmin'')
		\item możliwość pobierania raportów finansowych spółki w formacie PDF lub XLSX
		\item wyświetlanie tabeli zawierającej najważniejsze dane liczbowe na temat wybranego kwartału
		\item zapisywanie pobranych danych w pamięci podręcznej (cache), aby zmniejszyć liczbę zapytań do API
	\end{itemize}
	\section{Zdjęcia aplikacji}
	
	Poniżej przedstawiono screeny pokazujące 3 ekrany aplikacji
	
	\subsection{Ekran1: Wykres akcji}
	\begin{figure}[H]
		\centering
		\includegraphics[width=0.7\textwidth]{Obrazy/Ekran1.png}
		\caption{Ekran do wyświetlania wykresów akcji CD Projekt S.A. gdzie użytkownik może wybierać ich zakres}
		\label{fig:app1}
	\end{figure}
	
	\subsection{Ekran2: Wykres Google Trends}
	\begin{figure}[h]
		\centering
		\includegraphics[width=0.7\textwidth]{Obrazy/Ekran2.png}
		\caption{Ekran do wyświetlania wykresów Google Trends gdzie użytkownik może wybierać ich zakres i słowo kluczowe}
		\label{fig:app2}
	\end{figure}
    \newpage
	\subsection{Ekran3: Obsługa sprawozdań kwartalnych}
	\begin{figure}[h]
		\centering
		\includegraphics[width=0.7\textwidth]{Obrazy/Ekran3.png}
		\caption{Ekran do zarządzania sprawozdaniami kwartalnymi, gdzie użytkownik może je pobrać z wybranego kwartału}
		\label{fig:app3}
	\end{figure}
	
	\section{Struktura kodu}
	
	Aplikacja podzielona jest na moduły logiczne zgodnie z zasadą rozdziału odpowiedzialności.
	
	\subsection{Główna klasa aplikacji – \texttt{App}}
	Klasa ta inicjalizuje wszystkie ekrany aplikacji oraz zarządza ich wyświetlaniem. Po uruchomieniu programu użytkownik widzi ekran wykresu kursu akcji.
	
	\subsection{Ekran wykresów giełdowych – \texttt{Screen1}}
	Odpowiada za:
	
	\begin{itemize}
		\item wyświetlanie wykresu kursu,
		\item wybór zakresu danych (1 dzień, 1 tydzień, 1 miesiąc),
		\item podawanie zakresu użytkownika za pomocą kalendarza,
		\item aktualizację:
		\begin{itemize}
			\item bieżącej ceny akcji,
			\item minimalnej i maksymalnej ceny w zakresie.
		\end{itemize}
	\end{itemize}
	
	\subsection{Ekran Google Trends – \texttt{Screen2}}
	Udostępnia:
	
	\begin{itemize}
		\item wybór słowa kluczowego (np. CD Projekt, Wiedźmin),
		\item wygenerowanie wykresu popularności,
		\item odświeżanie danych,
		\item wykorzystanie cache w celu minimalizacji obciążenia API.
	\end{itemize}
	
	\subsection{Ekran sprawozdań – \texttt{Screen3}}
	Pozwala na:
	
	\begin{itemize}
		\item wybór kwartału z listy raportów,
		\item pobranie:
		\begin{itemize}
			\item raportu finansowego PDF,
			\item raportu XLSX z wynikami finansowymi,
			\item informacji prasowej w PDF.
		\end{itemize}
		\item wyświetlenie tabeli z najważniejszymi danymi na temat danego kwartału
	\end{itemize}
	
	\subsection{Moduł danych – \texttt{CDPdata}}
	Zawiera logikę:
	
	\begin{itemize}
		\item pobierania danych z Yahoo Finance,
		\item obsługi Google Trends poprzez PyTrends,
		\item zapisu i wczytywania cache w formacie JSON.
	\end{itemize}
	
	\subsection{Moduł wykresów – \texttt{CDPplot}}
	Odpowiada za renderowanie wykresów biblioteki matplotlib i osadzanie ich w komponentach interfejsu.
	
	\subsection{SplashScreen}
	Odpowiada za wyświetlanie startowego ekranu i ładowanie danych przed uruchomieniem aplikacji.

	\section {Opisy funkcji}
	
	\subsection*{CDPdata}
	\begin{itemize}
		\item \textbf{GetCdpData(period)} - pobiera dane o kursie z serwisu Yahoo Finance (yfinance) dla danego okresu
		\item \textbf{getCustomCDPData(start\_time, end\_time)} - pobiera dane o kursie dla wybranego przez użytkownika zakresu
		\item \textbf{GetCurrentPrice()} - pobiera i wyświetla aktualną cenę akcji
		\item \textbf{getMinMaxPrice(data)} - pobiera najniższą i najwyższą cenę w danym zakresie
		\item \textbf{ensure\_cache\_dir()} - upewnia się że istnieje folder plików JSON jeśli nie tworzy go
		\item \textbf{load\_cache()} - wczytuje cache a jeśli plik jest uszkodzony zwraca pusty
		\item \textbf{save\_cache(cache)} - zapisuje dane do pliku JSON
		\item \textbf{invalidate\_trends\_plot(keyword, period)} - usuwa cache przy odświeżaniu wykresów
		\item \textbf{timeframe\_for(period)} - zamienia nazwy okresów na te używane przez pytrends
		\item \textbf{df\_from\_entry(entry)} - tworzy DataFrame ze wczytanego cache
		\item \textbf{getTrendsData(period)} - wczytuje dane z pliku JSON a jeśli nie ma pobiera dane z Google Trends
		\item \textbf{choose\_folder()} - wyświetla okno wyboru folderu do jakiego ma zostać zapisany wybrany przez użytkownika plik
		\item \textbf{download\_file(url: str, default\_name: str)} - pobiera plik z internetu do wybranego przez użytkownika folderu
	\end{itemize}
	
	\subsection*{CDPplot}
	\begin{itemize}
		\item \textbf{CreateCdpPlot(frame, period)} - tworzy i wyświetla wykres cen
		\item \textbf{createCustomDataCdpPlot(frame, start\_date, end\_date, data)} - tworzy wykres dla wybranego przez użytkownika zakresu
		\item \textbf{createTrendsPlot(frame, period, data, keyword)} - tworzy wykres zainteresowania z Google Trends
	\end{itemize}
	
	\subsection*{SplashScreen}
	\begin{itemize}
		\item \textbf{\_\_init\_\_(self)} - ustawia wysokość szerokość i zawartość splash screena
		\item \textbf{center\_x(self, window\_width)}, \textbf{center\_y(self, window\_width)} - pomocnicze funkcje do ustawienia splash screena na środku ekranu
		\item \textbf{load\_data\_with\_delay(self)} - pobiera dane podczas splash screena i czeka co najmniej 2,5 sekundy
		\item \textbf{open\_main\_app(self, data)} - zamyka splash screen i otwiera główne okno aplikacji
	\end{itemize}
	
	\subsection*{CDPQuarter}
	\begin{itemize}
		\item \textbf{getQuarterTableData(selected\_quarter)} - wyciąga konkretne dane z pliku XLSX sprawozdania kwartalnego dla określonego kwartału i analogicznego kwartału z poprzedniego roku
	\end{itemize}
	
	\subsection*{App}
	\begin{itemize}
		\item \textbf{\_\_init\_\_(self, master, preloaded\_data, preloaded\_trends)} - ustawia wysokość szerokość i zawartość głównego okna aplikacji
		\item \textbf{showFrame(self, frame, text)} - zmienia widoczny ekran i nazwę aplikacji
	\end{itemize}
	
	\subsection*{Screen1}
	\begin{itemize}
		\item \textbf{\_\_init\_\_(self, master, preloaded\_data, preloaded\_trends)} - ustawia ramki okna wykresu akcji
		\item \textbf{showPlot(self, period)} - aktualizuje wykres po kliknięciu któregoś z przycisków
		\item \textbf{showCustomDatePlot(self)} - aktualizuje wykres po wpisaniu własnego zakresu
		\item \textbf{updatePriceLabel(self)} - aktualizuje cenę po kliknięciu przycisku
		\item \textbf{updateMinMaxLabels(self, data)} - aktualizuje min i max ceny po kliknięciu przycisku
		\item \textbf{showError(self, title, message, icon)} - pokazuje okienko z błędem
	\end{itemize}
	
	\subsection*{Screen2}
	\begin{itemize}
		\item \textbf{\_\_init\_\_(self, master, preloaded\_data, preloaded\_trends)} - ustawia ramki okna wykresów Google Trends
		\item \textbf{showTrendsPlot(self, period)} - tworzy wykres dla Google Trends z określonego zakresu
		\item \textbf{refreshTrends(self)} - odświeża dane używane do rysowania wykresu Google Trends
	\end{itemize}
	
	\subsection*{Screen3}
	\begin{itemize}
		\item \textbf{\_\_init\_\_(self, master, preloaded\_data, preloaded\_trends)} - ustawia ramki okna służącego do obsługi sprawozdań kwartalnych
		\item \textbf{onQuarterButtonClick(self)} - wyświetla tabelę z danymi po kliknięciu przycisku
		\item \textbf{showQuarterTable(self, extracted, selected\_quarter)} - tworzy tabelę z zebranych danych
		\item \textbf{download\_pdf(self)} - pobiera plik PDF
		\item \textbf{download\_xlsx(self)} - pobiera plik XLSX
		\item \textbf{download\_press\_pdf(self)} - pobiera dane prasowe w formacie PDF
		\item \textbf{showError(self, title, message, icon)} - pokazuje okienko z błędem
	\end{itemize}
	
	\section{Zagrożenia}
	
	\subsection{Błąd 429 Google Trends}
	Wynika z zbyt dużej liczby zapytań do serwisu Google Trends w krótkim czasie.\\
	\textbf{Rozwiązanie:} korzystanie z zapisanych danych w JSON (cache) oraz ograniczenie liczby zapytań
	
	\subsection{Blokowanie Yahoo Finance}
	Serwis Yahoo Finance wprowadza dzienne limity około 2000 zapytań na IP\\
	\textbf{Rozwiązanie:} ograniczanie częstotliwości zapytań, cache danych historycznych oraz stosowanie interwałów
	
	\subsection{Obsługa plików JSON (cache)}
	Możliwe błędy serializacji i deserializacji, np. uszkodzony plik JSON lub niepoprawny format danych\\
	\textbf{Rozwiązanie:} obsługa wyjątków przy wczytywaniu i zapisie JSON, atomowy zapis do pliku tymczasowego, backup uszkodzonego pliku
	
	\subsection{Path traversal w cache}
	Niepoprawnie sformatowane słowa kluczowe mogą prowadzić do zapisania/usunięcia plików poza katalogiem cache\\
	\textbf{Rozwiązanie:} sanitacja słów kluczowych (tylko litery, cyfry i podkreślenia), użycie \texttt{os.path.basename} oraz bezpieczne tworzenie nazw plików
	
	\subsection{Pobieranie plików z internetu}
	Pobieranie plików PDF/XLSX bez weryfikacji rozszerzenia, MIME type czy rozmiaru może prowadzić do pobrania złośliwego pliku\\
	\textbf{Rozwiązanie:} sprawdzanie typu i rozmiaru pliku, obsługa wyjątków, zapis w bezpiecznym folderze
	
	\subsection{Brak timeoutów w zapytaniach sieciowych}
	Funkcje \texttt{requests.get}, \texttt{yfinance.download} i Google Trends mogą blokować aplikację przy problemach sieciowych\\
	\textbf{Rozwiązanie:} ustawienie timeoutów i obsługa wyjątków przy pobieraniu danych
	
	\subsection{Duże rozmiary danych}
	Pobieranie dużych zbiorów danych (Google Trends, Yahoo Finance) może prowadzić do problemów z pamięcią\\
	\textbf{Rozwiązanie:} ograniczenie liczby pobieranych danych
	lub filtrowanie niepotrzebnych kolumn
	
	\subsection{Interfejs użytkownika}
	Brak pełnej kontroli nad ścieżką folderu wyboru w Tkinter może prowadzić do zapisu w systemowych katalogach\\
	\textbf{Rozwiązanie:} domyślny folder w katalogu aplikacji oraz walidacja ścieżki

	\section{Aktualizacje}
	
	\subsection*{27.10.2025}
	\begin{itemize}
		\item Podział aplikacji na moduły
		\item Dodanie funkcji do obsługi zakresów dat
		\item Nowe biblioteki: CtkMessagebox i tkcalendar
	\end{itemize}
	
	\subsection*{02.11.2025}
	\begin{itemize}
		\item Dodano pytrends
		\item Obsługa wykresów Google Trends
	\end{itemize}
	
	\subsection*{09.11.2025}
	\begin{itemize}
		\item Dodano obsługę zapisu trendów do JSON
		\item Podział aplikacji na 3 ekrany
	\end{itemize}
	
	\subsection*{16.11.2025}
	\begin{itemize}
		\item Dodano wykresy Google Trends dla Cyberpunk 2077 i Wiedźmin
		\item Dodano pobieranie sprawozdań finansowych CD Projekt SA
	\end{itemize}
	\subsection*{23.11.2025}
	\begin{itemize}
		\item Poprawienie dokumentacji
	\end{itemize}
	\subsection*{01.12.2025}
	\begin{itemize}
		\item Dodanie wyświetlania tabeli ze sprawozdania kwartalnego
		\item Przeprowadzenie testów jednostkowych i integracyjnych
	\end{itemize}
	\subsection*{08.12.2025}
	\begin{itemize}
		\item Przeprowadzenie testów wydajnościowych i funkcjonalnych
	\end{itemize}
	
	
\end{document}